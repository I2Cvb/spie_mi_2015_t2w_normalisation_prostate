%
% This are some examples using the acro package.

Example showing how plurals are easily handeled:
\\
first time: \ac{a} \\
second time: \ac{a} \\
short: \acs{a} \\
alternative: \aca{a} \\
first again: \acf{a} \\
long: \acl{a} \\
short plural: \acsp{a} \\
long plural: \aclp{a} \\

Here is an example of this \href{}{acro package}
using nested declarations and macro-acronyms calls, taken from
\href{http://tex.stackexchange.com/questions/135975/how-to-define-an-acronym-by-using-other-acronym-and-print-the-abbreviations-toge}{stackOverflow}.
\textbf{not working}
% \subsubsection*{nested acronyms example 1}

% \Ac{dna} is a very important molecule.  The virus xyz contains \dsdna.  Apart
% from that, \dna exists in almost all cells of the body. In most cases it is
% \dsdna.

% \subsubsection*{nested acronyms example 2}
% \acresetall

% The virus xyz contains \dsdna.  \Ac{dna} is a very important molecule.  Apart
% from that, \dna exists in almost all cells of the body. In most cases it is
% \dsdna.
