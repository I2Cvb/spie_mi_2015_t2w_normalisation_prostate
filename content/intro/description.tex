% include the figures path relative to the master file
\graphicspath{ {./content/intro/figures/} }

\section{Description}
\label{sec:descr}  % \label{} allows reference to this 

\Ac{cap} has been reported the second most frequently diagnosed cancer of men accounting for 13.6\%~\cite{ferlay2010estimates}. 
In United States, aside from skin cancer, \ac{cap} was considered to be the most commonly diagnosed cancer among men, implying that approximately 1 in 6 men will be diagnosed with \ac{cap} during their lifetime.
The \textit{American cancer society} also reported an estimated 233,000 new cases of prostate cancer in 2014~\cite{CancerFactsFigures2014}. 
To address these dramatic issues, more systematic screenings are organized through \ac{psa} test with further \ac{trus} biopsy if necessary.
However, these tests are unreliable or invasive and that is why further investigations using \ac{mri}-\ac{cad} are motivated. 
In the past decades, several \ac{cad} systems have been proposed in order to assist the radiologists with their diagnosis. 
These systems are usually designed as a sequential process consisting of four stages: pre-processing, segmentation, registration and classification.
As a pre-processing steps, image normalization is an important step of the chain. 
Normalization is a highly crucial step to overcome the inter-patient intensity variations occurring, enforce the repeatability, and achieve a robust classification~\cite{Lemaitre2015}.
However, little attention has been dedicated to the problem of normalization of T2W-\ac{mri} prostate images~\cite{Lemaitre2015}.
Artan\,\textit{et al.}~\cite{artan2010prostate,artan2009prostate} and Ozer\,\textit{et al.}~\cite{ozer2009prostate,ozer2010supervised} proposed to normalize the T2W-\ac{mri} images by computing the standard score (i.e., \textit{z-score}) of the \ac{pz} pixels such as: 
\begin{equation}
  I_{s}(x) = \frac{I_{r}(x) - \mu_{PZ}}{\sigma_{PZ}}, \forall x\in PZ
\end{equation}
\noindent Where, $I_{s}(x)$ and $I_{r}(x)$ are the standardized and the raw signal intensity, respectively, and $\mu_{PZ}$ and $\sigma_{PZ}$ are the mean and standard deviation of the \ac{pz} signal intensity. 
This transformation enforces the image \ac{pdf} to have a zero mean and a unit standard deviation.
Lv\,\textit{et al.}~\cite{lv2009computerized} used the method proposed by Nyul\,\textit{et al.}~\cite{nyul2000new}.
For a given patient, some specific landmarks (i.e., median and different percentiles) of the current \ac{pdf} match the same landmarks learned during a training phase from several patients.
Viswanath\,\textit{et al.}~\cite{viswanath2012central} used a variant of the previous method by segmenting first the image and keeping only the largest region to build the \ac{pdf}.
In this paper, we propose a method based on a Rician \textit{a priori} in order to normalize T2W-\ac{mri} prostate images. We emphasize the use of a Rician distribution before to present the methodology allowing to normalize the images. Both qualitative and quantitative results are given and compared with the previous stated methods.

% Some stuff that emac's colegues use
%%% Local Variables:
%%% mode: late
%%% TeX-master: "../../master.tex"
%%% End: \section{introduction}

