\section{Experiments}
\label{sec:exp}

{\color{red} Add the explanations regarding the piecewise-linear setup and percentiles. Add the implementation using the protoclass.}

The experiments are conducted on a subset of public multi-parametric \ac{mri} prostate dataset available at \texttt{http://visor.udg.edu/i2cvb/}~\cite{lemaitre2015boosting}.
This dataset was acquired from a cohort of patients with higher-than-normal level of \ac{psa}. The acquisition was performed using a 3T whole body \ac{mri} scanner (Siemens Magnetom Trio TIM, Erlangen, Germany) using sequences to obtain T2W-\ac{mri}. Aside of the \ac{mri} examinations, these patients also underwent a guided-biopsy. Finally, the dataset was composed of a total of 20 patients of which 18 patients had biopsy proven \ac{cap} and 2 patients were ``healthy'' with negative biopsies. In this study, our subset consists of 17 patients with \ac{cap}. The prostate organ as well as the prostate zones (i.e., \ac{pz}, \ac{cg}) and \ac{cap} were manually segmented by an experienced radiologist.