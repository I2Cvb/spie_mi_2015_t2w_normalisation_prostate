\section{Experiments}
\label{sec:exp}
The experiments are conducted on a subset of public multi-parametrci \ac{mri} prostate dataset~\cite{lemaitre2015boosting}.
This data was acquired from a cohort of patients with higher-than-normal level of PSA. The acquisition was performed using a 3T whole body \ac{mri} scanner (Siemens Magnetom Trio TIM, Erlangen, Germany) using sequences to obtain T2-W MRI (see Fig. 1a). Aside of the MRI examinations, these patients also underwent a guided-biopsy. Finally, the dataset was composed of a total of 20 patients of which 18 patients had biopsy proven \ac{cap} and 2 patients were ``healthy'' with negative biopsies. In this study our subset consists of 17 patients with \ac{cap}. The prostate organ as well as the prostate zones (i.e., \ac{pz}, \ac{cg}) and \ac{cap} were manually segmented by an experienced radiologist.