\section{Experiments}
\label{sec:exp}

\subsection{Data}

The experiments are conducted on a subset of public multi-parametric \ac{mri} prostate publicly available dataset\footnote{\url{http://visor.udg.edu/i2cvb/}}~\cite{lemaitre2015boosting}.
This dataset was acquired from a cohort of patients with higher-than-normal level of \ac{psa}. 
The acquisition was performed using a 3T whole body \ac{mri} scanner (Siemens Magnetom Trio TIM, Erlangen, Germany) using sequences to obtain T2W-\ac{mri}. 
Aside of the \ac{mri} examinations, these patients also underwent a guided-biopsy. 
Finally, the dataset was composed of a total of 20 patients of which 18 patients had biopsy proven \ac{cap} and 2 patients were ``healthy'' with negative biopsies. 
In this study, our subset consists of 17 patients with \ac{cap}. 
The prostate organ as well as the prostate zones (i.e., \ac{pz}, \ac{cg}) and \ac{cap} were manually segmented by an experienced radiologist.

\subsection{Implementation}

The different normalization methods are implemented in Python and publicly available in GitHub\footnote{\url{https://github.com/glemaitre/protoclass}}.
The normalization based on \ac{srsf} uses the implementation\footnote{\url{https://bitbucket.org/tetonedge/fdasrsf}} of Tucker\,\textit{et al.}~\cite{Tucker2013}.

\subsection{Parameters}

The model fitting for the Gaussian and Rician normalization is performed as a non-linear least squares problem, using Levenberg-Marquardt optimization.
The piecewise-linear normalization is performed using the following set of percentiles $s \in \{0, 5, 25, 50, 75, 95, 100 \}$ as landmarks.
In the \ac{srsf}-based normalization, the \acp{pdf} are smoothed using spline-based denoising method.
